\documentclass[]{article}

\usepackage{datetime}
\usepackage{color,array,graphics}
\usepackage{enumerate}
\usepackage{graphicx}
\graphicspath{ {images/} }

\setlength{\textheight}{8.5in}
\setlength{\textwidth}{6.5in}
\setlength{\oddsidemargin}{0in}
\setlength{\evensidemargin}{0in}
\voffset0.0in

\def\OR{\vee}
\def\AND{\wedge}
\def\imp{\rightarrow}
\def\math#1{$#1$}
\def\mand#1{$$#1$$}
\def\mld#1{\begin{equation}
#1
\end{equation}}
\def\eqar#1{\begin{eqnarray}
#1
\end{eqnarray}}
\def\eqan#1{\begin{eqnarray*}
#1
\end{eqnarray*}}
\def\cl#1{{\cal #1}}

\DeclareSymbolFont{AMSb}{U}{msb}{m}{n}
\DeclareMathSymbol{\N}{\mathbin}{AMSb}{"4E}
\DeclareMathSymbol{\Z}{\mathbin}{AMSb}{"5A}
\DeclareMathSymbol{\R}{\mathbin}{AMSb}{"52}
\DeclareMathSymbol{\Q}{\mathbin}{AMSb}{"51}
\DeclareMathSymbol{\I}{\mathbin}{AMSb}{"49}
\DeclareMathSymbol{\C}{\mathbin}{AMSb}{"43}

\begin{document}
\bf \Large Gabriel Maayan - FOCS Assignment 1

\section{DMC 1.3}
This comparison is wrong because net worth is a measurement of the cost of something, and GDP is a measurement of profit.
\section{DMC 2.5}
\begin{enumerate}[(b)]
\item \math{(A\cap\overline{(B\cup C)})\cup (A\cap B\cap C)}
\end{enumerate}

\section{DMC 2.14}
\begin{enumerate}[(a)]
\item \math{A\times B = \{ (1, a), (1, b), (1, c), (1, d), 
(2, a), (2, b), (2, c), (2, d),\newline (3, a), (3, b), (3, c), (3, d)\} 
\newline |A\times B| = 12 }
\item \math{B\times A = \{ (a, 1), (a, 2), (a, 3), (b, 1), (b, 2), (b, 3), 
\newline (c, 1), (c, 2), (c, 3), (c, 1), (c, 2), (c, 3), \} 
\newline |B\times A| = 12 }
\item \math{A\times A = \{ (1, 1), (1, 2), (1, 3), (2, 1), (2, 2), (2, 3), 
\newline (3, 1), (3, 2), (3, 3)\}
\newline |A\times A| = 9}
\item \math{B\times B = \{(a, a), (a, b), (a, c), (a, d), (b, a), (b, b), (b, c), (b, d), 
\newline (c, a), (c, b), (c, c), (c, d), (d, a), (d, b), (d, c), (d, d)\}
\newline |B\times B| = 16}
\end{enumerate}

\math{\\
.A\times B\times C = \{(a, b, c)  |  a\in A, b\in B, c \in C\}}

\section{DMC 3.33}
\math{
6: 1, 2, 3, 6
\newline 8: 1, 2, 4, 8
\newline 12: 1, 2, 3, 4, 6, 12
\newline 15: 1, 3, 5, 15
\newline 18: 1, 2, 3, 6, 9, 18
\newline 30: 1, 2, 3, 5, 6, 10, 15, 30
\newline 4: 1, 2, 4
\newline 9: 1, 3, 9
\newline 16: 1, 2, 4, 8, 16
\newline 25: 1, 5, 25
\newline 36: 1, 2, 3, 4, 6, 9, 12, 18, 36}


Let d(n) be the number of positive divisors of n.
\newline d(n) is odd if \math{n = x^2}, where x \math{\in \Z}.
\newline Otherwise d(n) is even.

\section{DMC 3.34}
\begin{enumerate}[(a)]
\item
\includegraphics[width=\linewidth]{infection}


\item It is not possible to fill the entire \math{6\times 6} grid from 5 original infections.
 6 original infections are able to fill the grid.
\newline For a \math{4\times 4} and \math{5\times 5} grid, you need at least 4 and 5 infections respectively.
\item For an \math{n\times n} grid, \math{n \in \N}, at least \math{n} original infections are needed to infect the entire grid.
\end{enumerate}

\section{DMC 1.2}
\begin{enumerate}[(f)]
\item An \math{8\times 8} grid gives 32 black squares and 32 white squares. If you remove one of each color, you have 31 of each. However, as long as there is an equal number of white and black squares, you will be able to cover the entire board. Each square removed shifts a domino over a square but since the dominoes are shifted twice, you can just remove one domino and cover the board.
\end{enumerate}


\end{document}